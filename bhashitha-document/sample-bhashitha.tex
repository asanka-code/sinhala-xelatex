\documentclass[12pt]{article}

\usepackage[a4paper,left=2.5cm, right=2.5cm, top=2.5cm, bottom=2.5cm]{geometry}
\usepackage{fontspec,xltxtra,xunicode}

\setmainfont[Extension=.ttf, Language=Sinhala, BoldFont=LBhashitaComplex, ItalicFont=LBhashitaComplex]{LBhashitaComplex}

% A fix for the upper-case A
\newfontfamily{\uctimes}[Scale=MatchUppercase]{FreeSerif}

\title{මාතෘකාව මෙතන ලියන්න}
\author{අසංක සායක්කාර}
\date{2017-07-16}

\begin{document}

\maketitle

\section{සිංහල}

මේ මම සිංහලෙන් සකස් කල ලිපියක්. මෙය සකස් කිරිඉමට මම \bf{ලේටෙක්} මෘදුකාංගය පාවිච්චි කලා. එය ඉතාම ප්‍රයෝජනවත් මෙවලමක් මට.

අසංකගේ නම ඉංග්‍රීසියෙන් ලියන්නේ {\uctimes A}sanka ලෙසයි. 
%අසංකගේ නම ඉංග්‍රීසියෙන් ලියන්නේ {\uctimes Asanka} ලෙසයි. 
වැඩිදුරටත් {\uctimes AAAAAA} අකුරු මෙහෙම ලියන්න පුළුවන්.
කැපිටල් {\uctimes A} අකුර මැදට එනකොට {\uctimes A}S{\uctimes A}NK{\uctimes A} මෙහෙම. ටිකක් කරදර වැඩේ හැබැයි.

\section{English}

Some English here and there.

{\uctimes A}sanka. {\uctimes A}BCDEFGHIJKLMNOPQRSTUVWXYZ.

\end{document}
